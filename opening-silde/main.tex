\documentclass[10pt,aspectratio=43]{beamer}
\usetheme{Berlin}

\usepackage{amsmath,bm,amsfonts,amssymb,enumerate,graphicx,animate,epsfig,bbm,calc,color,ifthen,capt-of,multimedia}
\usepackage{fancybox,xcolor,booktabs,colortbl}
\usepackage{physics}
\usepackage{../mycommand}%我惯用的命令,在本project中,存储在本文件的父文件夹
\graphicspath{{../figures/}}

\usefonttheme{professionalfonts}
\usepackage[UTF8,fontset=none,scheme=chinese]{ctex}
\setmainfont{Times}
\setsansfont{Arial}
\setmonofont{Consolas}
\setCJKmainfont{SimSun}[BoldFont=SimHei,ItalicFont=KaiTi]
\setCJKsansfont{SimHei}
\setCJKmonofont{Microsoft YaHei}
\usepackage{unicode-math}
\setmathfont{XITS Math}

\usepackage{../beamercolorthemesustech}%效果同\usecolortheme{sustech},但\usecolortheme不支持自由选择文件路径,故调用底层的\usepackage命令
\definecolor{mygreen}{rgb}{0,0.6,0}
\definecolor{mymauve}{rgb}{0.58,0,0.82}
\definecolor{mygray}{gray}{.9}
\definecolor{mypink}{rgb}{.99,.91,.95}
\definecolor{mycyan}{cmyk}{.3,0,0,0}

\usepackage[ruled,linesnumbered]{algorithm2e}
\usepackage{verbatim,listings}
\lstset{ %
	backgroundcolor=\color{white},   % choose the background color
	basicstyle=\footnotesize\ttfamily,     % size of fonts used for the code
	columns=fullflexible,
	breaklines=true,                 % automatic line breaking only at whitespace
	captionpos=b,                    % sets the caption-position to bottom
	tabsize=4,
	commentstyle=\color{mygreen},    % comment style
	escapeinside={\%*}{*)},          % if you want to add LaTeX within your code
	keywordstyle=\color{blue},       % keyword style
	stringstyle=\color{mymauve}\ttfamily,     % string literal style
	numbers=left, 
	%	frame=single,
	rulesepcolor=\color{red!20!green!20!blue!20},
	% identifierstyle=\color{red},
	language=c
}

\setbeamertemplate{caption}[numbered]%添加图片的编号
\setbeamertemplate{sidebar right}{}%去掉默认添加的navigarion symbols
\setbeamercovered{transparent}%使未点出来的文字呈现透明,默认0.15透明度
\beamerdefaultoverlayspecification{}

\usepackage[backend=biber,bibstyle=gb7714-2015,citestyle=verbose]{biblatex}
\addbibresource{../reference.bib}

%题目,作者,学校,日期
\title{多连接延迟混杂系统的\\自触发间歇周期脉冲控制}
\subtitle{\fontsize{9pt}{14pt}\textbf{毕业论文开题报告}}
\author{答辩人:杨徵羽 \quad 指导教师:李文学}
\institute{哈尔滨工业大学(威海)理学院}
\date{\today}

%学校Logo
%\pgfdeclareimage[height=0.5cm]{sustech-logo}{sustech-logo.pdf}
%\logo{\pgfuseimage{sustech-logo}\hspace*{0.3cm}}

\AtBeginSection[]
{
	\begin{frame}<beamer>
	\frametitle{\textbf{目录}}
	\tableofcontents[currentsection]
\end{frame}
}

\begin{document}

\frame{\titlepage}

\section[目录]{}   %目录
\begin{frame}{目录}
\tableofcontents
\end{frame}

\section{课题背景}
\begin{frame}{随机系统}
\alert{随机耦合系统}在模拟现实系统中得到了广泛应用,如传染病传播\footcite{2012modelling}、社交网络\footcite{1999small}

\alert{马尔可夫切换}用于描述系统受到的无规则结构性变化(具有马尔可夫切换的系统也称混杂系统)

\alert{时间延迟}用于描述系统传输的延迟对响应的影响,且这一延迟可能是随时间变化的

含马尔可夫切换和时间延迟的系统的研究如文献\footcite{2015global, 2018stabilization}
\end{frame}

\begin{frame}{多连接}
值得注意的是,上述文献中考虑的系统都只有单连接,然而现实世界中更广泛存在的是\alert{多连接系统}
\begin{block}{例子:交通网络}
将城市视为一个节点,城市间的交通方式视为节点的连接,则公路、铁路、空运等多种交通方式对应着节点之间的多连接
\end{block}
因此有必要研究多连接延迟混杂系统(HDMS),已有的文献包括:
\begin{itemize}
\item 文献\footcite{2020stochastic}研究了多连接延迟混杂随机系统在非周期自适应间歇控制下的均方指数稳定性
\item 文献\footcite{2020synchronization}研究了多连接系统的同步稳定分布
\end{itemize}
\end{frame}

\begin{frame}{间歇脉冲控制}
为使系统同步,控制学家提出了多种多样的控制方法

相比连续控制,\alert{不连续控制}具有更强的机动性,分为:
\begin{itemize}
\item \alert{脉冲控制}\footcite{2019review}:只需在一系列脉冲时刻传递信息

但由于实现脉冲控制要在瞬间改变状态,在工程中不易实现
\item \alert{间歇控制}\footcite{2021improved}:只在预先决定的非零时间间隔中施加控制,更易实现
\end{itemize}

近期有学者提出了\alert{间歇脉冲控制} (IIC)\footcite{2013intermittent, 2016intermittent},将脉冲控制和间歇控制相结合,进一步降低了控制频率,降低成本
\end{frame}

\begin{frame}{自触发控制}
需要指出,上述文献中的间歇脉冲控制是\alert{时间触发}的,即控制器在预定的时钟信号下工作,在应用上有局限性
\begin{block}{例子:网络通信\footcite{2003on}}
网络通信系统中,各个节点的时钟和网络丢包不一致,无法直接施加时间触发控制
\end{block}
实际上,我们可以采用事件触发或自触发控制,其触发与否取决于信号与预设阈值间的关系,而非固定的时间周期

其中,\alert{自触发控制} (STC)\footcite{2020self}由即时的计算确定触发时刻,当处于触发时刻时施加控制
\end{frame}

\begin{frame}{周期自触发控制与间歇脉冲控制的结合}
\begin{itemize}
\item 近年有学者提出了\alert{周期自触发控制} (PSTC)\footcite{2021synchronization}\footcite{2019sampling},在确定触发时刻时只需周期性地测量系统状态,降低了采样时间并避免了时间触发的缺陷
\item 其中前者还将脉冲控制结合进周期自触发控制中,提高了控制的性能
\item 本课题更进一步,将更先进的间歇脉冲控制结合进周期自触发控制中,得到\alert{自触发间歇周期脉冲控制} (PSTIIC)
\end{itemize}

\end{frame}
\begin{frame}{研究内容与创新点}
研究内容
\begin{itemize}
\item 系统:多连接延迟混杂系统(HDMS)
\item 控制:自触发间歇周期脉冲控制(PSTIIC)
\item 预期性质:同步性
\end{itemize}
创新点
\begin{itemize}
\item 系统具有多连接、时变延迟和马尔可夫切换,涵盖范围较为广泛
\item 将间歇脉冲控制结合与周期自触发控制相结合,提高控制能力
\item 这一新式控制方法可应用于含有时变延迟和马尔可夫切换的蔡氏电路
\end{itemize}
\end{frame}

\begin{frame}{}
\begin{center}
\begin{minipage}{1\textwidth}
\setbeamercolor{mybox}{fg=white, bg=black!50!blue}
\begin{beamercolorbox}[wd=0.70\textwidth, rounded=true, shadow=true]{mybox}
\LARGE \centering 请各位老师批评指正
\end{beamercolorbox}
\end{minipage}
\end{center}
\end{frame}

\end{document}
